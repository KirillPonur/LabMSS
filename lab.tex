\def\labauthors{Карусевич А.А., Понур К.А.}
\def\labgroup{440}
\def\labnumber{1}
\def\labtheme{Колебания механических систем \\[0.2em] с распределенными параметрами}
\def\shortlabtheme{Колебания пластин и стержней}
\def\department{Кафедра акустики}
\input{text/diss}
\begin{document}
%!TEX root = ../fet.tex
\begin{titlepage}
\begin{center}
% \vspace{-3em}
{\small\textsc{Нижегородский государственный университет имени Н.\,И. Лобачевского}}
\vskip 2pt \hrule \vskip 3pt
{\small\textsc{Радиофизический факультет}}

\vfill


{{\large Отчет по лабораторной работе №\labnumber}\vskip 12pt {\LARGE \bfseries \labtheme}}

	
\vspace{2cm}
{\large Работу выполнили студенты \\[-0.25em] \labgroup\  группы радиофизического факультата \\[0.5em] {\Large \bfseries \labauthors}}

% \vspace{0.5cm}
% {e-mail: sfg180@yandex.ru}

% \vspace{2cm}

\end{center}

\vfill
	
% \begin{flushright}
% 	{Выполнили студенты 430 группы\\ \labauthor}%\vskip 12pt Принял:\\ Менсов С.\,Н.}
% \end{flushright}
	
% \vfill
	
\begin{center}
	{Нижний Новгород, \labstartdate\ -- \today}
\end{center}

\end{titlepage}
\tableofcontents
\newpage


\addcontentsline{toc}{section}{Введение}
\section*{Введение}
\vspace{-0.5em}
В настоящей работе исследуются продольные колебания стержней и поперечные колебания пластин с помощью резонансного метода -- возбуждаются колебания на резонансных частотах. Под пластиной понимается упругое трехмерное тело, один размер которого много меньше двух других, а под стержнем -- тело, у которого один размер больше двух других. При этом пластины и стержни можно считать тонкими, если длина волны велика по сравнению с их толщиной.

В эксперименте используются три стержня (из алюминия, стали и оргстекла, все длины 394 мм) и две металлические пластины (толщины 0.63 мм и 1.16 мм)
\vspace{-0.5em}

\section*{Колебания стержней}
Механические свойства однородных и изотропных упругих тел, обладающих потерями, могут быть описаны следующими параметрами: модулем Юнга Е, модулем сдвига, коэффициентом вязкости $\eta$ и плотностью $\rho$.

Рассмотрим продольные колебания, возбужденные посредством приложения периодической силы $F_0e^{-i\omega t}$, действующей в направлении оси стержня. Уравнение продольных колебаний в тонком абсолютно упругом стержне, имеющем по всей длине постоянное сечение (без затухания):
\begin{equation}
	\xi(x,t)=\frac{F_0e^{-i\omega t}cosk(l-x)}{kES\sin(\frac{\omega l}{c})},
\end{equation}
где $l$ длина стержня, k и c соответственно волновое число и фазовая скорость звука в отсутствие потерь, s - единица поперечного сечения.

Видно, что в идеальном стержне без затухания устанавливается чисто стоячая волна, амплитуда смещения которой вдоль стержня распределена по косинусоиде и сильно зависит от частоты вынуждающей силы. Резонанс наблюдается при $\omega=\frac{\pi n c}{l}$ при частотах $f_n=nc/2l$.

В случае малых потерь резонанс наступает почти при тех же частотах, что и а стержне без потерь. Возбуждение продольных колебаний в стержне дает возможность определить модуль Юнга и коэффициент вязкости:
\begin{equation}
	E=\rho c^2, \eta = \frac{E}{Q\omega},
\end{equation}
где Q - добротность стержня, $\omega$ - резонансная частота. 

\section*{Изгибные колебания пластин}
Для малых прогибов тонкой пластины требуются слабые внешние усилия, приложенные к ее поверхности. Эти усилия значительно меньше, чем внутренние напряжения, которые возникают внутри деформированной пластины благодаря имеющимся в ней растяжениям и сжатиям. 

Колебания пластины, происходящие с собственными частотами, носят название нормальных колебаний (мод). Каждая мода колебаний характеризуется двойным индексом mn. Число m соответствует порядку бесселевской функции и совпадает с числом узлов окружностей, за исключением граничной. Число n соответствует порядковому номеру решения характеристического уравнения и совпадает с числом узловых диаметров без единицы. 

При колебаниях с образованием узловых колец и диаметров поверхность пластины разбивается на зоны, разделенные узловыми линиями, причем колебания в любой зоне происходят в противофазе с соседними зонами. При низшей частоте $f_{01}$ вся поверхность пластины колеблется с одной фазой. 

Под действием на пластину силой, изменяющейся по гармоническому закону, в пластине возбуждаются вынужденные колебания. Если частота возбуждающей силы соответствует частоте одной из собственных мод колебаний пластины, то наступает механический резонанс. 

\paragraph{Установка с пластинами.} Металлическая пластина закреплена в станке, в котором размещен электромагнитный возбудитель колебаний. Возбудитель можно перемещать вдоль диаметра пластинки и наблюдать при этом различные типы колебаний. На пластине рассыпается тонкий слой песка, который при резонансе собирается в узлах колебаний пластины. Полученные картины называются картинами Хладни и позволяют определить конкретную моду колебаний.

% \begin{figure}[H]
% 	\centering
% 	\includegraphics[width=\textwidth]{fig/view}
% 	\vspace{-1em}
% 	\caption{Лабораторная установка. 1 -- осциллограф ОСУ-10А, 2 -- генератор сигналов UNI-T UTG9010C, 3 -- измерительный модуль, подключенный в режиме усилителя входного сигнала, 4 -- мультиметр APPA-201N, 5,6 -- источники питания GPS-3030D (в принципиальной схеме -- $E_2$, $E_1$).}
% 	\label{fig:1}
% \end{figure}

\newpage

\section{Резонансные кривые продольных колебаний}
\subsection{Стальной стержень}

\begin{figure}[H]
	\centering
	\includegraphics[scale=1.5]{fig/steel_afc.pdf}
	\vspace{-1em}
	\caption{Семейство переходных характеристик}
	\label{fig:2}
\end{figure}
Из графика снятой АЧХ нашли параметры резонансной кривой: резонансную частоту
$f_0=6610\text{ Гц}$ и ширину на уровне 0.7 $\Delta f_{0.7} = 0.48\text{ Гц}$. 
Частота хорошо согласуется с теоретическим значением первой моды
\begin{equation}
	f_1 = \frac{1\cdot c}{2l} = \frac{5210 \text{ м}\cdot\text{с}^{-1}}{2\cdot 0.394\text{ м}}=6611 \text{ Гц}
\end{equation}
Добротность колебательной системы
\begin{equation}
	Q = \frac{f_0}{\Delta f_{0.7}} \approx 13770
\end{equation}
Исходя из табличного значения модуля Юнга для стали $E=0.20\cdot10^{12}$ Па, нашли вязкость стержня  \cite[стр. 10]{met}:
\begin{equation}
	\eta = \frac{\gamma E}{\omega_0} = \frac{ E}{Q 2\pi f_0} = \frac{\Delta f_{0.7} E}{2\pi f_0^2} = 340 \text{ кг}\cdot\text{м}^{-1}\cdot\text{с}^{-1}
\end{equation}

\subsection{Алюминиевый стержень}
\begin{figure}[H]
	\centering
	\includegraphics[scale=1.5]{fig/al_afc}
	\caption{Принципиальная схема}
	\label{fig:chem1}
\end{figure}

Из графика снятой АЧХ нашли параметры резонансной кривой: резонансную частоту
$f_0=6458.6\text{ Гц}$ и ширину на уровне 0.7 $\Delta f_{0.7} \approx 1\text{ Гц}$. 
Частота согласуется с теоретическим значением первой моды
\begin{equation}
	f_1 = \frac{1\cdot c}{2l} = \frac{5140 \text{ м}\cdot\text{с}^{-1}}{2\cdot 0.394\text{ м}}=6522 \text{ Гц}
\end{equation}
Добротность колебательной системы
\begin{equation}
	Q = \frac{f_0}{\Delta f_{0.7}} \approx 6522
\end{equation}
Исходя из табличного значения модуля Юнга для алюминия $E=0.07\cdot10^{12}$ Па, нашли вязкость стержня  \cite[стр. 10]{met}:
\begin{equation}
	\eta = \frac{\gamma E}{\omega_0}= \frac{ E}{Q 2\pi f_0} = \frac{\Delta f_{0.7} E}{2\pi f_0^2}= 270 \text{ кг}\cdot\text{м}^{-1}\cdot\text{с}^{-1}
\end{equation}
\subsection{Латунный стержень}
\begin{figure}[H]
	\centering
	\includegraphics[scale=1.5]{fig/lat_afc}
	\caption{Принципиальная схема}
	\label{fig:chem11}
\end{figure}

Из графика снятой АЧХ нашли параметры резонансной кривой: резонансную частоту
$f_0=3497.5 \text{ Гц}$ и ширину на уровне 0.7 $\Delta f_{0.7} \approx 0.4\text{ Гц}$. 
Частота согласуется с теоретическим значением первой моды
\begin{equation}
	f_1 = \frac{1\cdot c}{2l} = \frac{2830 \text{ м}\cdot\text{с}^{-1}}{2\cdot 0.394\text{ м}}=3591 \text{ Гц}
\end{equation}
Добротность колебательной системы
\begin{equation}
	Q = \frac{f_0}{\Delta f_{0.7}} \approx 8743
\end{equation}
Исходя из табличного значения модуля Юнга для алюминия $E=0.071\cdot10^{12}$ Па, нашли вязкость стержня  \cite[стр. 10]{met}:
\begin{equation}
	\eta = \frac{\gamma E}{\omega_0}= \frac{ E}{Q 2\pi f_0} = \frac{\Delta f_{0.7} E}{2\pi f_0^2}= 369 \text{ кг}\cdot\text{м}^{-1}\cdot\text{с}^{-1}
\end{equation}


\subsection{Стержень из оргстекла}

\begin{figure}[H]
	\centering
	\includegraphics[scale=1.5]{fig/glass_afc}
	\caption{Резонансная кривая колебаний в стержне из оргстекла}
	\label{fig:chem1}
\end{figure}

Из графика снятой АЧХ нашли параметры резонансной кривой: резонансную частоту
$f_0=2420\text{ Гц}$ и ширину на уровне 0.7 $\Delta f_{0.7} = 96\text{ Гц}$. 
Частота согласуется с теоретическим значением первой моды
\begin{equation}
	f_1 = \frac{1\cdot c}{2l} = \frac{2040 \text{ м}\cdot\text{с}^{-1}}{2\cdot 0.394\text{ м}}=2588 \text{ Гц}
\end{equation}
Добротность колебательной системы
\begin{equation}
	Q = \frac{f_0}{\Delta f_{0.7}} \approx 25
\end{equation}
Исходя из табличного значения модуля Юнга для оргстекла $E=0.005\cdot10^{12}$ Па, нашли вязкость стержня  \cite[стр. 10]{met}:
\begin{equation}
	\eta = \frac{\gamma E}{\omega_0}= \frac{ E}{Q 2\pi f_0} = \frac{\Delta f_{0.7} E}{2\pi f_0^2} = 13150 \text{ кг}\cdot\text{м}^{-1}\cdot\text{с}^{-1}
\end{equation}




\newpage
\subsection{Экспериментальное значение модуля Юнга}

В предыдущих пунктах значение модуля Юнга бралось apriori, а согласованность проверялась сравнением первых мод. Однако, можно рассчитать уточненное значение модуля Юнга, рассчитав в обратном порядке скорость звука в продольном стержне, исходя из экспериментальных данных:
\begin{equation}
	c=\frac{2l f_n}{n}=2lf_n=
	\left\{
	\begin{aligned}
		5089 \text{ м}\cdot\text{с}^{-1},&\quad \text{алюминий}\\
		1097 \text{ м}\cdot\text{с}^{-1},&\quad \text{оргстекло}\\
		2756 \text{ м}\cdot\text{с}^{-1},&\quad \text{латунь}\\
		5208 \text{ м}\cdot\text{с}^{-1},&\quad \text{сталь}
	\end{aligned}
	\right.
\end{equation}
Считая известными плотности, найдем модуль Юнга:
\begin{equation}
	E = \rho c^2=
	\left\{
	\begin{aligned}
		0.069\cdot 10^{12}\text{ Па},&\quad \text{алюминий}\\
		0.0014\cdot 10^{12}\text{ Па},&\quad \text{оргстекло}\\
		0.0006\cdot 10^{12}\text{ Па},&\quad \text{латунь}\\
		0.211\cdot 10^{12}\text{ Па},&\quad \text{сталь}
	\end{aligned}
	\right.
\end{equation}
Тогда уточнённое значение вязкости для стали и алюминия изменится в пределах погрешности измерений, а для оргстекла
\begin{equation}
	\eta_{new}\approx 3700 \text{ кг}\cdot\text{м}^{-1}\cdot\text{с}^{-1}
\end{equation}

Возникает вопрос о причине такого сильного расхождения. Это можно объяснить тем, что вычисляюся величины, опирающиеся на эксперимент, т.е. снятую резонансную кривую, которая характеризует не колебания в стержне, но и характеристики колебательной системы в целом, в том числе, приемника/передатчика.

\newpage

\section{Поперечные колебания круглых пластин}

В данном эксперименте изучались т.н. фигуры Хладни, образующиеся из скоплений частиц песка на колеблющейся пластине в узлах колебаний. Фигуры был получены на различных модах колебаний в пределах 0.5--1.5 кГц для двух пластин различной толщины при двух положениях излучателя.

\subsection{Пластина 0.63 мм}
\subsubsection{Излучатель в центре}
\begin{figure}[H]
	\centering
	\includegraphics[scale=1.5]{fig/63_c.pdf}
\end{figure}
\subsubsection{Излучатель смещен}
\begin{figure}[H]
	\centering
	\includegraphics[scale=1.5]{fig/63_b.pdf}
\end{figure}
\subsection{Пластина 0.80 мм}
\subsubsection{Излучатель в центре}
\begin{figure}[H]
	\centering
	\includegraphics[scale=1.5]{fig/80_c.pdf}
\end{figure}
\subsubsection{Излучатель смещен}
\begin{figure}[H]
	\centering
	\includegraphics[scale=1.5]{fig/80_b.pdf}
\end{figure}

\subsection{Пластина 1.16 мм}
\subsubsection{Излучатель в центре}
\begin{figure}[H]
	\centering
	\includegraphics[scale=1.5]{fig/116_c.pdf}
\end{figure}
\subsubsection{Излучатель смещен}
\begin{figure}[H]
	\centering
	\includegraphics[scale=1.5]{fig/116_b.pdf}
\end{figure}

Можно отметить, что при смещении излучателя частота основного тона повышается. В каждом из экспериментов были сняты основной тон и один и более обертонов. Наиболее высокий обретон, который удалось получить -- $f_{31}$ для пластины 0.63 мм.

\subsection{Теоретические значения частот}
Собственные частоты изгибных колебаний пластины определяются формулой
\begin{equation}
	\omega_{mn} = \frac{\pi^2 H}{a^2}\beta_{mn}^2 \qty|\frac{E}{3\rho_s (1-\nu^2)}|^{\frac12}
\end{equation}

Где значения $\beta_{mn}$ порождаются решением уравнений относительно функций Бесселя и не связаны с характеристиками установки. $\beta_{mn}$ можно считать известными.

Так как некоторые константы были не известны, можно поступить следующим образом: брать основной тон из эксперимента и пытаться рассчитать по формуле обертона. Например, для основного тона $f_{01}=165$ Гц рассчитаем обертона $f_{02,03}$:
\begin{equation}
	f_{02} = 3.309 f_{01} = 545 \text{ Гц},\qquad 
	f_{03} = 2.234 f_{02} = 1217 \text{ Гц} 
\end{equation}
В эксперименте же наблюдались частоты 567 и 1226 Гц. Завышение теоретических значений можно объяснить наличием диссипации: так, по аналогии, учет диссипации для колебаний в LC-контуре приводит к уменьшению резонансной частоты.
\newpage
\addcontentsline{toc}{section}{Заключение}
\section*{Заключение}
В настоящей работе были изучены линейные теории двумерных колебательных систем с распределенными параметрами (пластин и стержней); проведен ряд экспериментов с пластинами и стержнями. 

Для стержней были получены резонансные кривые, рассчитаны добротность, модуль Юнга и коэффициент вязкости для каждого стержня. Следует отметить различие в ширинах резонансных кривых (и, как следствие, добротностях) для стали и оргстекла. Для оргстекла ширина резонансной кривой составляет несколько сотен Герц (и добротность $\sim10$), а для стали ширина кривой несколько Герц (а добротность $\sim 10^4$). Такая большая разница связана с различиями в структурах оргстекла и стали. Оргстекло -- аморфный материал с высокой (по сравнению со сталью) вязкостью, поэтому колебания в нем распространяются хуже, чем в стали. 

В экспериментах с пластинами были получены фигуры Хладни, определены моды колебаний для двух положений возбудителя: по центру пластины и сдвинутым относительно центра.

% \newpage
\begin{thebibliography}{}
  \bibitem{mss} Гурбатов С.Н. Лекции по механике сплошных сред на радиофизическом факультете 2018/2019. -- 106 с.
  
  \bibitem{met} Горская Н.\,В., Курин В.\,В. и др. Колебания механической системы с распределенными параметрами: колебания стержней. Н.Новгород: ННГУ, 1995. -- 13 с.
  
  % \bibitem{lit3} Ландау Л.Д., Лифшиц Е.М. Любой том. М.: Физматлит, 2003.
\end{thebibliography}

\end{document}
